% !TEX encoding = UTF-8
% !TEX program = pdflatex
% !TEX root = relazione-MEMOC.tex
% !TeX spellcheck = it_IT

\section{Conclusioni}

Il modello CPLEX è risultato nettamente superiore all'algoritmo genetico, principalmente perché le istanze sulle quali è stato eseguito erano di dimensione ridotte. 
Nel caso pratico ci sono schede con molti più fori e quindi trovare una soluzione ottima potrebbe richiedere molto più di 10 minuti.

D'altro canto l'algoritmo genetico ora come ora, non è una grande alternativa in quanto il gap dalla soluzione ottima è considerevole. 
In ogni caso, la certezza dell'ottimalità con l'algoritmo genetico non si può avere, ma le modifiche suggerite potrebbero fornire soluzioni migliori rispetto a quelle attuali.

C'è poi un'altra considerazione da fare riguardo la distribuzione dei fori nella scheda, che nel caso pratico sono tra loro raggruppati in insiemi di fori vicini.
Si è cercato di imitare ciò con le istanze pseudo-casuali e si è osservato che al crescere del numero dei fori, sia il modello CPLEX che l'algoritmo genetico, fanno più fatica a risolvere tali istanze.
Questo può essere dovuto al fatto che il modello CPLEX ``\textit{perda tempo}'' ad ottimizzare lo spostamento all'interno dei vari gruppi, mentre per l'algoritmo genetico vengono creati più minimi locali ai quali converge la popolazione.
Pertanto un algoritmo euristico alternativo potrebbe prima risolvere il problema principale, ovvero trovare il ciclo ottimo che permette di visitare tutti i vari gruppi di fori da effettuare e poi ottimizzare il percorso interno di ciascuno gruppo.
Così facendo si scompone il problema in più sotto-problemi con meno variabili e quindi più veloci da risolvere. 
Inoltre, è ragionevole pensare che la parte più critica sia l'ordine in cui vengono visitati i vari gruppi di fori, perché questi possono essere più sparsi, mentre l'ordine in cui vengono effettuati i fori che appartengono ad uno stesso insieme risulta meno importante perché sono tra loro vicini.
Si può quindi progettare un approccio ibrido, che usa un modello esatto per calcolare il ciclo ottimo di visita dei vari gruppi e un'euristica per determinare il percorso di foratura interno ai vari insiemi. 



