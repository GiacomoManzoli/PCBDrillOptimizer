% !TEX encoding = UTF-8
% !TEX program = pdflatex
% !TEX root = relazione-MEMOC.tex
% !TeX spellcheck = it_IT

\section{Algoritmo Genetico}\label{sec:genetico}

Come meta-euristica ad-hoc si è scelto di implementare un algoritmo genetico seguendo le indicazioni presenti nelle dispense del corso.

\subsection{Scelte progettuali}

Gli algoritmi genetici lasciano molte possibilità di scelta al progettatore e le scelte effettuate influenzano notevolmente l'efficacia dell'algoritmo.

Nel determinare i vari componenti si è cercato di progettare un algoritmo bilanciato, che parta da delle soluzioni buone, ma che converga lentamente grazie alle mutazioni e alla selezione di Montecarlo.

\subsubsection{Codifica delle soluzioni}

Per la codifica delle soluzioni si è scelto di adottare la stessa utilizzata per CPLEX.
Viene quindi utilizzato un array di lunghezza $N+1$, dove $N$ è il numero di nodi da visitare o fori da effettuare, e rappresenta la sequenza di visita. La dimensione dell'array è di $N+1$ perché viene aggiunto un ultimo elemento sempre fisso a $0$, per imporre il vincolo che la trivella ritorni al punto di partenza.
Allo stesso modo è imposto il vincolo che il primo elemento dell'array sia 0, in modo che nodo di partenza sia sempre quello e che coincida con il nodo finale.

% TODO inserire immagine 

\subsubsection{Generazione della popolazione iniziale}

La popolazione iniziale viene creata con delle soluzioni generate in modo pseudo-greedy.
Ovvero, ogni soluzione viene generata incrementalmente a partire dal nodo di partenza, andando a scegliere come nodo successivo un nodo qualsiasi tra quelli ancora da visitare.
La scelta del nodo viene fatta a caso, dando una maggiore probabilità di essere scelti ai nodi migliori

\subsubsection{Funzione di fitness}

Come funzione di fitness per gli individui è stata utilizzata la funzione obiettivo, ovvero il costo del cammino descritto dalla soluzione.

\subsubsection{Operatore di selezione}

La selezione delle soluzioni da riprodurre viene fatta secondo un \textit{torneo-K}. Vengono scelti casualmente dalla popolazione $K$ individui, con $K = \text{POP\_SIZE}/10$ e tra questi viene scelto il miglior candidato per partecipare alla riproduzione.
Il processo viene quindi eseguito due volte in modo da scegliere i due genitori.

\subsubsection{Crossover}

Il crossover viene effettuato in modo uniforme utilizzando i due genitori precedentemente scelti, dando maggior probabilità di esser trasmessi ai geni del genitore migliore.
La combinazione dei geni viene effettuata costruendo un nuovo cammino a partire dagli archi presenti nei cammini dei due genitori.

Sia $succ(x, G)$ il nodo successivo al nodo $x$ nel cammino della soluzione $G$, pertanto nella soluzione $G$ sarà presente l'arco $(x, succ(x,G))$ e siano $G_1$ e $G_2$ i due genitori della nuova soluzione.

Si ha quindi che la costruzione del nuovo cammino partirà dal nodo 0 e pertanto il secondo nodo del cammino verrà scelto casualmente tra $succ(0,G_1)$ e $succ(0,G_2)$.
Al passo successivo, l'ultimo nodo inserito nel nuovo cammino sarà un nodo $y$ e pertanto la scelta del nodo su cui spostarsi sarà tra $succ(y,G_1)$ o $succ(y, G_2)$.
Il procedimento viene ripetuto finché non sarà completato il ciclo, tornando al nodo 0.

Durante la costruzione del figlio possono capitare alcuni casi particolari:

\begin{itemize}
	\item Uno dei due possibili successori è già presente nel cammino. In questo caso viene scelto l'altro.
	\item Entrambi i nodi fanno già parte del cammino. Il questo caso come successore viene scelto il nodo migliore.
	\item Il nodo finale del cammino deve essere il nodo 0. Quindi l'ultimo arco viene scelto forzatamente in modo che sia verso il nodo 0.
\end{itemize}

Da notare che per come sono gestiti questi casi particolari non possono essere generate soluzioni non valide.

\subsubsection{Mutazione}

\`E stata prevista la possibilità che durante l'evoluzione delle popolazione, alcune soluzioni subiscano una mutazione.

Una mutazione consiste nel rimescolare l'ordine di visita dei nodi interni del cammino. In questo modo la mutazione viene fatta velocemente e non invalida la soluzione, perché il primo e l'ultimo nodo vistato sarà sempre il nodo 0.

\subsubsection{Sostituzione della popolazione}

La sostituzione delle popolazione viene effettuata generando prima un numero $R$ di individui proporzionale alla dimensione della popolazione. Dopodiché la popolazione viene riportata alla dimensione di partenza, selezionando con il metodo di Montecarlo $N$ soluzioni tra le $N+R$ disponibili.

\subsubsection{Criterio di stop}

Come criterio d'arresto è stato utilizzato un time limit che può essere specificato all'avvio dell'algoritmo.

La scelta è ricaduta su questa condizione d'arresto perché così risulta più semplice effettuare il confronto con CPLEX e perché l'altro criterio provato, ovvero fermare l'algoritmo dopo $k$ iterazioni che non hanno migliorato la miglior soluzione, richiedeva troppo tempo d'esecuzione a causa del metodo di sostituzione della popolazione.
Infatti, tra un'iterazione e l'altra può essere scartata anche la soluzione migliore e quindi l'iterazione successiva può risultare migliorativa anche se in realtà non lo è.

L'altro criterio d'arresto preso in considerazione è stato un limite sulle iterazioni, ma facendo le varie prove si è osservato che è preferibile impostare un tempo limite fisso piuttosto che il numero massimo di iterazioni.

\subsection{Parametri dell'algoritmo e processo di ottimizzazione}

L'algoritmo così implementato richiede che siano specificati i seguenti parametri:

\begin{itemize}
	\item\texttt{POPULATION\_SIZE}: dimensione della popolazione;
	\item\texttt{MUTATION\_RATE}: probabilità di mutazione:
	\item\texttt{GROWTH\_RATIO}: soluzioni generate ad ogni iterazione;
	\item\texttt{TIME\_LIMIT}: tempo limite per l'esecuzione dell'algoritmo.
\end{itemize}

\subsubsection{Esperimenti per l'ottimizzazione dei parametri}

Prima di confrontare l'approccio genetico con CPLEX è stata eseguita una leggera ottimizzazione dei parametri.

Si sono quindi provate le varie possibili combinazioni per la risoluzione di varie istanze di dimensioni diverse (50, 100, 150 punti) e generate sia in modo casuale che pseudo-casuale.
Le varie prove sono state poi ripetute più volte ed è stata effettuata una media dei risultati.

I valori per i parametri che sono stati provati sono:

\begin{itemize}
	\item \texttt{POPULATION\_SIZE}: 100, 250, 500;
	\item \texttt{MUTATION\_RATE}: 0.01, 0.05, 0.1;
	\item \texttt{GROWTH\_RATIO}: 1.1, 1.5, 2;
	\item \texttt{TIME\_LIMIT}: 1 minuto.
\end{itemize}

% Todo aggiungere esiti


