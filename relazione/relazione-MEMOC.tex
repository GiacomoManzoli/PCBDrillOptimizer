% !TEX encoding = UTF-8
% !TEX program = pdflatex
% !TEX root = relazione-MEMOC.tex
% !TeX spellcheck = it_IT
\documentclass[a4paper, 11pt]{article} % Font size (can be 10pt, 11pt or 12pt) and paper size (remove a4paper for US letter paper)
\usepackage[italian]{babel}    							% Lingua italiana
\usepackage[margin=.9in]{geometry}             % Imposta i margini del documento

\usepackage[T1]{fontenc} % Required for accented characters
\usepackage[mathletters]{ucs}    % Caratteri matematici come UTF8
\usepackage[utf8,utf8x]{inputenc}      % Ancora utf8

\usepackage{eurosym}                %simbolo dell'euro
\usepackage[usenames,dvipsnames,svgnames,table]{xcolor}
\usepackage{listings}
\input{listings}

% Inserisco questi 2 package altrimenti su windows non compila! 
\usepackage[T1]{fontenc}
\usepackage{lmodern}

%\tightlist per compatibilità con pandoc
\providecommand{\tightlist}{%
	\setlength{\itemsep}{0pt}\setlength{\parskip}{0pt}}


\usepackage[labelfont=bf]{caption}

\usepackage[protrusion=true,expansion=true]{microtype} % Better typography
\usepackage{graphicx} % Required for including pictures
\usepackage{wrapfig} % Allows in-line images

\usepackage{multirow} 
\usepackage{subfig}
\usepackage{hyperref}
\usepackage{placeins}
\usepackage[scaled=0.8]{sourcecodepro}
\usepackage{hyperref}                   % collegamenti ipertestuali
\hypersetup{
	%hyperfootnotes=false,
	%pdfpagelabels,
	%draft,	% = elimina tutti i link (utile per stampe in bianco e nero)
	colorlinks=true,
	linktocpage=true,
	pdfstartpage=1,
	pdfstartview=FitV,
	% decommenta la riga seguente per avere link in nero (per esempio per la stampa in bianco e nero)
	%colorlinks=false, linktocpage=false, pdfborder={0 0 0}, pdfstartpage=1, pdfstartview=FitV,
	breaklinks=true,
	pdfpagemode=UseNone,
	pageanchor=true,
	pdfpagemode=UseOutlines,
	plainpages=false,
	bookmarksnumbered,
	bookmarksopen=true,
	bookmarksopenlevel=1,
	hypertexnames=true,
	pdfhighlight=/O,
	%nesting=true,
	%frenchlinks,
	urlcolor=Cerulean,
	linkcolor=RoyalBlue,
	citecolor=Cerulean,
	%pagecolor=RoyalBlue,
	%urlcolor=Black, linkcolor=Black, citecolor=Black, %pagecolor=Black,
	pdfsubject={},
	pdfkeywords={},
	pdfcreator={pdfLaTeX},
	pdfproducer={LaTeX}
}

\usepackage[colorinlistoftodos,prependcaption]{todonotes} %todo

\usepackage{amsmath}
\usepackage{mathtools}

\usepackage{float}
\usepackage{algorithm}
\usepackage{algpseudocode} % https://en.wikibooks.org/wiki/LaTeX/Algorithms#Typesetting_using_the_algorithmicx_package
\usepackage{amssymb}  %$\mathbb{N}$ per il simbolo dei numeri naturali 

\usepackage{enumerate} % permette di personalizzare enumerate

\usepackage{xmpincl}	%Aggiunge metadati sulla licenza CC
\usepackage{xspace}

\usepackage{bussproofs}

\makeatletter
\renewcommand\@biblabel[1]{\textbf{#1.}} % Change the square brackets for each bibliography item from '[1]' to '1.'
\renewcommand{\@listI}{\itemsep=0pt} % Reduce the space between items in the itemize and enumerate environments and the bibliography

\renewcommand{\maketitle}{ % Customize the title - do not edit title and author name here, see the TITLE block below
	\begin{flushright} % Right align
		{\LARGE\@title} % Increase the font size of the title
		
		\vspace{50pt} % Some vertical space between the title and author name
		
		{\large\@author} % Author name
		\\\@date % Date
		
		\vspace{100pt} % Some vertical space between the author block and abstract
	\end{flushright}
}

%% breakablealgorithm http://tex.stackexchange.com/questions/33866/algorithm-tag-and-page-break
\makeatletter
\newenvironment{breakablealgorithm}
{% \begin{breakablealgorithm}
	\begin{center}
		\refstepcounter{algorithm}% New algorithm
		\hrule height.8pt depth0pt \kern2pt% \@fs@pre for \@fs@ruled
		\renewcommand{\caption}[2][\relax]{% Make a new \caption
			{\raggedright\textbf{\ALG@name~\thealgorithm} ##2\par}%
			\ifx\relax##1\relax % #1 is \relax
			\addcontentsline{loa}{algorithm}{\protect\numberline{\thealgorithm}##2}%
			\else % #1 is not \relax
			\addcontentsline{loa}{algorithm}{\protect\numberline{\thealgorithm}##1}%
			\fi
			\kern2pt\hrule\kern2pt
		}
	}{% \end{breakablealgorithm}
	\kern2pt\hrule\relax% \@fs@post for \@fs@ruled
\end{center}
}
\makeatother

\makeatletter % trattino con punto sopra
\newcommand{\dotminus}{\mathbin{\text{\@dotminus}}}

\newcommand{\@dotminus}{%
	\ooalign{\hidewidth\raise1ex\hbox{.}\hidewidth\cr$\m@th-$\cr}%
}
\makeatother

\DeclarePairedDelimiter{\ceil}{\lceil}{\rceil}
\DeclarePairedDelimiter{\floor}{\lfloor}{\rfloor}


\let\oldtext\text



%----------------------------------------------------------------------------------------
% TITLE
%----------------------------------------------------------------------------------------
\begin{document}
%**************************************************************
% Frontespizio 
%**************************************************************
\begin{titlepage}
	
	\begin{center}
		
		\begin{LARGE}
			\textbf{Università degli Studi di Padova}\\
		\end{LARGE}
		
		\vspace{10pt}
		
		\begin{Large}
			\textsc{Dipartimento di Matematica}\\
		\end{Large}
		
		\vspace{10pt}
		
		\begin{large}
			\textsc{Corso di Laurea Magistrale in Informatica}\\
		\end{large}
		
		\vspace{30pt}
		\begin{figure}[htbp]
			\begin{center}
				\includegraphics[height=6cm]{immagini/logo-unipd}
			\end{center}
		\end{figure}
		\vspace{10pt} 
		
		\begin{LARGE}
			\begin{center}
				\textbf{Relazione esercitazioni laboratorio}\\
			\end{center}
		\end{LARGE}
		
		\vspace{10pt} 
		
		\vspace{60pt} 
		
		\begin{large}
			%\begin{flushleft}
			%	\textit{Relatore}\\ 
			%	\vspace{5pt} 
		 	%	Prof. CEP
			%\end{flushleft}
			
			\begin{flushright}
				\textit{Autore}\\ 
				\vspace{5pt} 
				Giacomo Manzoli 1130822
			\end{flushright}
		\end{large}
		
		\vspace{160pt}
		
		\line(1, 0){338} \\
		\begin{normalsize}
			\textsc{Anno Accademico 2016-2017}
		\end{normalsize}
		
	\end{center}
\end{titlepage} 

%----------------------------------------------------------------------------------------


%----------------------------------------------------------------------------------------
%	DOCUMENT HEADER
%----------------------------------------------------------------------------------------


	%----------------------------------------------------------------------------------------
	% ABSTRACT AND KEYWORDS
	%----------------------------------------------------------------------------------------
	
	%\renewcommand{\abstractname}{Summary} % Uncomment to change the name of the abstract to something else
	
	\clearpage
	\tableofcontents
	
	%\hspace*{3,6mm}\textit{Keywords:} lorem , ipsum , dolor , sit amet , lectus % Keywords
	
	%\vspace{30pt} % Some vertical space between the abstract and first section
	
	%----------------------------------------------------------------------------------------
	% ESSAY BODY
	%----------------------------------------------------------------------------------------
	\clearpage
	% !TEX encoding = UTF-8
% !TEX program = pdflatex
% !TEX root = relazione-MEMOC.tex
% !TeX spellcheck = it_IT

\section{Introduzione}

L'obiettivo della prima parte del progetto è quello di implementare un modello in CPLEX e di provarlo in modo da trovare la dimensione massima del problema che permette una risoluzione esatta entro un certo limite di tempo, mentre la seconda parte prevede l'implementazione di un algoritmo meta-euristico ad-hoc per il problema e per poi confrontarlo con il modello CPLEX.

La sezione §\ref{sec:cplex} contiene la descrizione dell'implementazione del modello CPLEX con i relativi test, mentre la sezione §\ref{sec:genetico} contiene la descrizione dell'algoritmo genetico implementato e un confronto delle prestazioni rispetto a quelle ottenute da CPLEX.

\subsection{Generazione delle istanze}

Prima di implementare i vari algoritmi è stato necessario creare delle istanze per il problema.
\`E stato quindi creato uno script in Python in grado di generare delle istanze casuali a partire da un numero di nodi o fori.

Dato che il problema ha delle caratteristiche specifiche, ovvero visto che si tratta di schede perforate, è ragionevole assumere che i fori seguano un certo pattern.
Pertanto lo script è stato sviluppato in modo che possa generare anche delle istanze pseudo-casuali, ovvero delle istanze in cui ci sono blocchi di punti che compaiono vicini tra loro, raggruppati in rettangoli e a coppie.

In entrambi i casi, una volta generati i punti, le distanze sono state calcolate utilizzando la distanza euclidea.

\begin{figure}[htbp]
	\centering
	\includegraphics[width=.5\textwidth]{immagini/scheda.png}
	\caption{Esempio di scheda perforata presa come riferimento per la generazione delle istanze pseudo-casuali.}
\end{figure}	% Introduzione
	% !TEX encoding = UTF-8
% !TEX program = pdflatex
% !TEX root = relazione-MEMOC.tex
% !TeX spellcheck = it_IT

\section{CPLEX} \label{sec:cplex}

Il modello CPLEX è stato implementato come specificato nella consegna della prima esercitazione con dei particolari accorgimenti per rendere il codice più comprensibile e mantenibile:

\begin{itemize}
	\item Le informazioni relative al problema e ad una sua possibile soluzione sono state modellate con due classi \texttt{Problem} e \texttt{Solution}. Inoltre, tutta la logica di risoluzione è stata incapsulata nella classe \texttt{CPLEXSolver}.
	\item Durante la dichiarazione delle variabili viene costruita una mappa di supporto per rendere più agevole l'utilizzo delle variabili all'interno dei vincoli. Viene inoltre limitato il numero di variabili, evitando di definire le variabili $x_{ii}$ e $y_{ii}$.
	\item I vincoli vengono definiti uno alla volta, in modo da semplificare la sintassi di definizione.
\end{itemize}

\subsection{Definizione delle variabili}

Le variabili in CPLEX, una volta create, vengono memorizzate in sequenza all'interno di un array interno del risolutore e l'unico modo per riferirsi ad una variabile è mediante la sua posizione nell'array interno.

Per riferirsi più facilmente alle variabili, viene quindi creata una matrice di dimensione $N\times N$ che associa il ``\textit{nome della variabile}'' alla sua posizione interna nel risolutore.

\begin{lstlisting}[language=C++, caption=Creazione delle variabili $x_{ij}$]
// xMap[i][j] è una matrice N x N
for (int i = 0; i < N; ++i) {
	for (int j = 0; j < N; ++j) {
		if (i == j) continue;
		char htype = 'I';
		double obj = 0.0;
		double lb = 0.0;
		double ub = CPX_INFBOUND;
		snprintf(name, NAME_SIZE, "x_%d,%d", nodes[i], nodes[j]);
		char* xname = &name[0];
		CHECKED_CPX_CALL( CPXnewcols, env, lp, 1, &obj, &lb, &ub, &htype, &xname );
		xMap[i][j] = createdVars;
		createdVars++;
	}
}
\end{lstlisting}

\noindent La mappatura del nome viene fatta nella riga 12 del frammento: \texttt{createdVars} è una variabile che tiene traccia del numero di variabili che sono state create nel risolutore e quindi la prossima variabile aggiunta avrà come indice interno il valore di \texttt{createdVars}. Il valore dell'indice viene quindi memorizzato nella mappa e poi incrementato, in modo che alla successiva iterazione dei ciclo, questo sia ancora corretto.

Nello stesso frammento di codice è possibile osservare come \textbf{non} vengano create le variabili $x_{ii}$, questo perché quando i due indici sono uguali, l'\texttt{if} in riga 4 blocca l'esecuzione del corpo. 
Questa scelta è stata fatta perché tale variabile non è significativa per il problema, in quanto scegliere di spostare la trivella lungo l'arco $(i,i)$ equivale al lasciare ferma la trivella.

Quanto riportato è stato effettuato anche per le variabili $y_{ij}$.

\subsection{Definizione dei vincoli}

CPLEX permette di definire più vincoli con una sola istruzione, tuttavia la notazione per sfruttare questa possibilità è poco pratica da usare, in quanto richiede che gli indici delle variabili e i corrispettivi coefficienti vengano passati come una matrice sparsa, linearizzata in un vettore.
Se invece viene creato un vincolo alla volta, non c'è la necessità di gestire la matrice sparsa, in quanto questa è composta da una sola riga e quindi può essere considerata come vettore. 
Detto in altre parole, non è necessario utilizzare il vettore \texttt{rmatbeg} per tenere traccia dell'inizio delle varie righe, dato che essendoci un'unica riga, questa inizierà alla posizione 0 degli array utilizzati per definire il vincolo.

\begin{lstlisting}[language=C++, caption=Esempio di creazione di una serie di vincoli]
// Vincoli sul flusso in ingresso
for (int j = 0; j < N; ++j){
	std::vector<int> varIndex(N-1);
	std::vector<double> coef(N-1);
	int idx = 0;
	// Recupero gli indici dalla mappa delle varibaili
	for (int i = 0; i < N; ++i) {
		if (i==j) continue;
		varIndex[idx] = yMap[i][j];
		coef[idx] = 1;
		idx++;
	}
	char sense = 'E';
	double rhs = 1;
	snprintf(name, NAME_SIZE, "in_%d",j+1);
	char* cname = (char*)(&name[0]);

	int matbeg = 0;
	CHECKED_CPX_CALL( CPXaddrows, env, lp, 
						0, // Numero di variabili da creare
						1, // Numero di vincoli da creare
						varIndex.size(), // Numero di variabili nel vincolo con coeff != 0
						&rhs, // Parte destra del vincolo
						&sense, // Senso
						&matbeg, // 0 perché creo un solo vincolo
						&varIndex[0], // Inizio dell'array con gli indici delle variabili
						&coef[0], // Inizio dell'array con i coefficienti delle variabili
						NULL,  // Nomi per le nuove variabili
						&cname // Nome del vincolo
					);
}
\end{lstlisting}

\noindent Dal codice sopra riportato si può osservare come la chiamata della riga 19 definisce solamente un vincolo, pertanto per generare tutti vincoli del tipo

$$
\sum\limits_{j : (i,j) \in A} y_{ij} = 1 \quad \forall j \in N
$$

\noindent è necessario effettuare un iterazione esterna con il ciclo \texttt{for} di riga 2.

Questo modo di definire i vincoli potrebbe essere leggermente meno efficiente, dato che effettua più chiamate ai metodi del risolutore, ma l'impatto sulle prestazioni rimane comunque basso, dato che i vincoli vengono creati solamente all'inizializzazione del modello e il tempo necessario è molto inferiore rispetto a quello necessario per l'ottimizzazione.		% Descrizione della struttura del programma
	% !TEX encoding = UTF-8
% !TEX program = pdflatex
% !TEX root = relazione-MEMOC.tex
% !TeX spellcheck = it_IT

\section{Algoritmo Genetico}\label{sec:genetico}

Come meta-euristica ad-hoc si è scelto di implementare un algoritmo genetico seguendo le indicazioni presenti nelle dispense del corso.

\subsection{Scelte progettuali}

Gli algoritmi genetici lasciano molte possibilità di scelta al progettatore e le scelte effettuate influenzano notevolmente l'efficacia dell'algoritmo.

Nel determinare i vari componenti si è cercato di progettare un algoritmo bilanciato, che parta da delle soluzioni buone, ma che converga lentamente grazie alle mutazioni e alla selezione di Montecarlo.

\subsubsection{Codifica delle soluzioni}

Per la codifica delle soluzioni si è scelto di adottare la stessa utilizzata per CPLEX.
Viene quindi utilizzato un array di lunghezza $N+1$, dove $N$ è il numero di nodi da visitare o fori da effettuare, e rappresenta la sequenza di visita. La dimensione dell'array è di $N+1$ perché viene aggiunto un ultimo elemento sempre fisso a $0$, per imporre il vincolo che la trivella ritorni al punto di partenza.
Allo stesso modo è imposto il vincolo che il primo elemento dell'array sia 0, in modo che nodo di partenza sia sempre quello e che coincida con il nodo finale.

% TODO inserire immagine 

\subsubsection{Generazione della popolazione iniziale}

La popolazione iniziale viene creata con delle soluzioni generate in modo pseudo-greedy.
Ovvero, ogni soluzione viene generata incrementalmente a partire dal nodo di partenza, andando a scegliere come nodo successivo un nodo qualsiasi tra quelli ancora da visitare.
La scelta del nodo viene fatta a caso, dando una maggiore probabilità di essere scelti ai nodi migliori

\subsubsection{Funzione di fitness}

Come funzione di fitness per gli individui è stata utilizzata la funzione obiettivo, ovvero il costo del cammino descritto dalla soluzione.

\subsubsection{Operatore di selezione}

La selezione delle soluzioni da riprodurre viene fatta secondo un \textit{torneo-K}. Vengono scelti casualmente dalla popolazione $K$ individui, con $K = \text{POP\_SIZE}/10$ e tra questi viene scelto il miglior candidato per partecipare alla riproduzione.
Il processo viene quindi eseguito due volte in modo da scegliere i due genitori.

\subsubsection{Crossover}

Il crossover viene effettuato in modo uniforme utilizzando i due genitori precedentemente scelti, dando maggior probabilità di esser trasmessi ai geni del genitore migliore.
La combinazione dei geni viene effettuata costruendo un nuovo cammino a partire dagli archi presenti nei cammini dei due genitori.

Sia $succ(x, G)$ il nodo successivo al nodo $x$ nel cammino della soluzione $G$, pertanto nella soluzione $G$ sarà presente l'arco $(x, succ(x,G))$ e siano $G_1$ e $G_2$ i due genitori della nuova soluzione.

Si ha quindi che la costruzione del nuovo cammino partirà dal nodo 0 e pertanto il secondo nodo del cammino verrà scelto casualmente tra $succ(0,G_1)$ e $succ(0,G_2)$.
Al passo successivo, l'ultimo nodo inserito nel nuovo cammino sarà un nodo $y$ e pertanto la scelta del nodo su cui spostarsi sarà tra $succ(y,G_1)$ o $succ(y, G_2)$.
Il procedimento viene ripetuto finché non sarà completato il ciclo, tornando al nodo 0.

Durante la costruzione del figlio possono capitare alcuni casi particolari:

\begin{itemize}
	\item Uno dei due possibili successori è già presente nel cammino. In questo caso viene scelto l'altro.
	\item Entrambi i nodi fanno già parte del cammino. Il questo caso come successore viene scelto il nodo migliore.
	\item Il nodo finale del cammino deve essere il nodo 0. Quindi l'ultimo arco viene scelto forzatamente in modo che sia verso il nodo 0.
\end{itemize}

Da notare che per come sono gestiti questi casi particolari non possono essere generate soluzioni non valide.

\subsubsection{Mutazione}

\`E stata prevista la possibilità che durante l'evoluzione delle popolazione, alcune soluzioni subiscano una mutazione.

Una mutazione consiste nel rimescolare l'ordine di visita dei nodi interni del cammino. In questo modo la mutazione viene fatta velocemente e non invalida la soluzione, perché il primo e l'ultimo nodo vistato sarà sempre il nodo 0.

\subsubsection{Sostituzione della popolazione}

La sostituzione delle popolazione viene effettuata generando prima un numero $R$ di individui proporzionale alla dimensione della popolazione. Dopodiché la popolazione viene riportata alla dimensione di partenza, selezionando con il metodo di Montecarlo $N$ soluzioni tra le $N+R$ disponibili.

\subsubsection{Criterio di stop}

Come criterio d'arresto è stato utilizzato un time limit che può essere specificato all'avvio dell'algoritmo.

La scelta è ricaduta su questa condizione d'arresto perché così risulta più semplice effettuare il confronto con CPLEX e perché l'altro criterio provato, ovvero fermare l'algoritmo dopo $k$ iterazioni che non hanno migliorato la miglior soluzione, richiedeva troppo tempo d'esecuzione a causa del metodo di sostituzione della popolazione.
Infatti, tra un'iterazione e l'altra può essere scartata anche la soluzione migliore e quindi l'iterazione successiva può risultare migliorativa anche se in realtà non lo è.

L'altro criterio d'arresto preso in considerazione è stato un limite sulle iterazioni, ma facendo le varie prove si è osservato che è preferibile impostare un tempo limite fisso piuttosto che il numero massimo di iterazioni.

\subsection{Parametri dell'algoritmo e processo di ottimizzazione}

L'algoritmo così implementato richiede che siano specificati i seguenti parametri:

\begin{itemize}
	\item\texttt{POPULATION\_SIZE}: dimensione della popolazione;
	\item\texttt{MUTATION\_RATE}: probabilità di mutazione:
	\item\texttt{GROWTH\_RATIO}: soluzioni generate ad ogni iterazione;
	\item\texttt{TIME\_LIMIT}: tempo limite per l'esecuzione dell'algoritmo.
\end{itemize}

\subsubsection{Esperimenti per l'ottimizzazione dei parametri}

Prima di confrontare l'approccio genetico con CPLEX è stata eseguita una leggera ottimizzazione dei parametri.

Si sono quindi provate le varie possibili combinazioni per la risoluzione di varie istanze di dimensioni diverse (50, 100, 150 punti) e generate sia in modo casuale che pseudo-casuale.
Le varie prove sono state poi ripetute più volte ed è stata effettuata una media dei risultati.

I valori per i parametri che sono stati provati sono:

\begin{itemize}
	\item \texttt{POPULATION\_SIZE}: 100, 250, 500;
	\item \texttt{MUTATION\_RATE}: 0.01, 0.05, 0.1;
	\item \texttt{GROWTH\_RATIO}: 1.1, 1.5, 2;
	\item \texttt{TIME\_LIMIT}: 1 minuto.
\end{itemize}

% Todo aggiungere esiti


		% Problemi implementativi
	% !TEX encoding = UTF-8
% !TEX program = pdflatex
% !TEX root = relazione-MEMOC.tex
% !TeX spellcheck = it_IT

\section{Conclusioni}

Il modello CPLEX è risultato nettamente superiore all'algoritmo genetico, principalmente perché le istanze sulle quali è stato eseguito erano di dimensione ridotte. 
Nel caso pratico ci sono schede con molti più fori e quindi trovare una soluzione ottima potrebbe richiedere molto più di 10 minuti.

D'altro canto l'algoritmo genetico ora come ora, non è una grande alternativa in quanto il gap dalla soluzione ottima è considerevole. 
In ogni caso, la certezza dell'ottimalità con l'algoritmo genetico non si può avere, ma le modifiche suggerite potrebbero fornire soluzioni migliori rispetto a quelle attuali.

C'è poi un'altra considerazione da fare riguardo la distribuzione dei fori nella scheda, che nel caso pratico sono tra loro raggruppati in insiemi di fori vicini.
Si è cercato di imitare ciò con le istanze pseudo-casuali e si è osservato che al crescere del numero dei fori, sia il modello CPLEX che l'algoritmo genetico, fanno più fatica a risolvere tali istanze.
Questo può essere dovuto al fatto che il modello CPLEX ``\textit{perda tempo}'' ad ottimizzare lo spostamento all'interno dei vari gruppi, mentre per l'algoritmo genetico vengono creati più minimi locali ai quali converge la popolazione.
Pertanto un algoritmo euristico alternativo potrebbe prima risolvere il problema principale, ovvero trovare il ciclo ottimo che permette di visitare tutti i vari gruppi di fori da effettuare e poi ottimizzare il percorso interno di ciascuno gruppo.
Così facendo si scompone il problema in più sotto-problemi con meno variabili e quindi più veloci da risolvere. 
Inoltre, è ragionevole pensare che la parte più critica sia l'ordine in cui vengono visitati i vari gruppi di fori, perché questi possono essere più sparsi, mentre l'ordine in cui vengono effettuati i fori che appartengono ad uno stesso insieme risulta meno importante perché sono tra loro vicini.
Si può quindi progettare un approccio ibrido, che usa un modello esatto per calcolare il ciclo ottimo di visita dei vari gruppi e un'euristica per determinare il percorso di foratura interno ai vari insiemi. 



    % Conclusioni
	\appendix						
	% !TEX encoding = UTF-8
% !TEX program = pdflatex
% !TEX root = relazione-MEMOC.tex
% !TeX spellcheck = it_IT

\section{Informazioni sul codice e sulla consegna}

Il codice consegnato è suddiviso in varie sotto-directory:

\begin{itemize}
	\item \texttt{common}: contiene le classi comuni, utilizzate sia dal modello CPLEX che dall'algoritmo genetico.
	\item \texttt{cplex}: contiene le classi relative all'implementazione del modello CPLEX.
	\item \texttt{ga}: contiene le classi relative all'implementazione dell'algoritmo genetico.
\end{itemize}

Il codice è stato poi strutturato in modo simile ad una libreria: per entrambe le parti del progetto c'è una classe che incapsula l'algoritmo risolutivo e che espone un metodo \texttt{solve()}.
Pertanto non è presente un vero e proprio file principale dell'applicazione.

Vengono però forniti vari file \texttt{.cpp} che contengo degli esempi di utilizzo della libreria e che corrispondo ai programmi utilizzati per eseguire i test.
Viene anche fornito un \texttt{main.cpp} che risolve in entrambi i modi un'istanza di un problema generata casualmente. Tale programma può essere compilato utilizzando il comando \texttt{make}.
All'interno del \texttt{Makefile} sono definite anche le regole per compilare gli altri sorgenti.


\subsection{Possibili problemi di compilazione}

Il programma \texttt{main.cpp}, e di conseguenza tutto il codice prodotto, compilano correttamente sui computer del laboratorio con \texttt{g++ v5.4}.
Tuttavia potrebbero esserci dei problemi nel linking delle librerie di CPLEX. In questo caso può essere necessario correggere i \textit{filepath} delle librerie CPLEX all'interno definiti all'interno del \texttt{Makefile}.
		% Come eseguire il codice
	%----------------------------------------------------------------------------------------
	%	CONTENT
	%----------------------------------------------------------------------------------------
	
	
\end{document}